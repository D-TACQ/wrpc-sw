This section tries to identify all the possible ways the White Rabbit PTP Core can
fail. The structure of each error description is the following:
\begin{itemize}[leftmargin=0pt]
	\item [] \underline{Severity}: describes how critical is the fault. Currently
		we distinguish two severity levels:
		\begin{packed_items}
			\item WARNING - means that despite the fault the synchronization
				functionality was not affected so the WRPC behaves correctly in the WR
				network.
			\item ERROR - means that the fault is critical and most probably a WRPC
				misbehaves.
		\end{packed_items}
	\item [] \underline{Mode}: for timing failures, it describes which modes are
		affected. Possible values are:
		\begin{packed_items}
			\item \emph{Slave} - the WR PTP Core synchronizes to another WR device.
			\item \emph{Grand Master} - the WR node (WR PTP Core) at the top of the
				synchronization hierarchy. It is synchronized to an external clock (e.g.
				GPS, Cesium) and provides timing to other WR/PTP devices.
			\item \emph{Master} - the WR node (WR PTP Core) at the top of the
				synchronization hierarchy. It provides timing to other WR/PTP devices
				but runs from a local oscillator (not synchronized to an external
				clock).
			\item \emph{all} - any WR PTP Core can be affected regardless the timing
				mode.
		\end{packed_items}

	\item [] \underline{Description}: What the problem is about, how important it
		is and what are the effects if it occurs.
	\item [] \underline{SNMP objects}: Which SNMP objects should be monitored to
		detect the failure. These are objects from the \texttt{WR-WRPC-MIB}.
	\item [] \underline{Error/Warning condition}: condition that should be checked
		at the SNMP manager's side to detect given problem.
  \item [] \underline{Action}: list of actions that should be performed in case
    given error/warning is reported.
\end{itemize}

\subsection{Timing error}
\label{sec:timing_fail}
As a timing error we define the WR PTP Core not being able to synchronize its
local time to the WR Master (if WRPC runs in the slave mode), or not being able
to provide correct WR time to the rest of the WR network (if WRPC runs in the
master mode).

\noindent This section contains the list of faults leading to a timing error.

\subsubsection{\bf PTP/PPSi went out of \texttt{TRACK\_PHASE}}
		\label{fail:timing:ppsi_track_phase}
		\begin{pck_descr}
			\item [] \underline{Severity}: ERROR
			\item [] \underline{Mode}: \emph{Slave}
			\item [] \underline{Description}:\\
				If the \emph{PTP/PPSi} WR servo goes out of the \texttt{TRACK\_PHASE}
				state, this means something bad has happened and the node lost the
				synchronization to its Master.
			\item [] \underline{SNMP objects}:\\
				{\footnotesize
				\snmpadd{WR-WRPC-MIB::wrpcPtpServoStateN}\\
				\snmpadd{WR-WRPC-MIB::wrpcPtpServoStateErrCnt} }
			\item [] \underline{Error condition}:\\
				{\footnotesize
				\texttt{wrpcPtpServoStateErrCnt != wrpcPtpServoStateErrCnt\_prev} }
      \item [] \underline{Action}:
        \begin{pck_proc}
        \item Dump state
        \item Check the status of the WR Switch which is the timing master for a
          given WR node.
        \item If the WR Switch did not report any problems, restart the WR Node.
        \item If the problem persists replace the WR Node hardware with a new
          unit.
        \item If the problem persists, please notify WR experts.
        \end{pck_proc}
		\end{pck_descr}

\subsubsection{\bf Offset jump not compensated by Slave}
		\label{fail:timing:offset_jump}
		\begin{pck_descr}
			\item [] \underline{Severity}: ERROR
			\item [] \underline{Mode}: \emph{Slave}
			\item [] \underline{Description}:\\
				This may happen if the Master resets its WR time counters (e.g. because
				it lost the link to its Master higher in the hierarchy or to external
				clock), but the WR Slave does not follow the jump.
				\item [] \underline{SNMP objects}:\\
				{\footnotesize
				\snmpadd{WR-WRPC-MIB::wrpcPtpClockOffsetPsHR}\\
				\snmpadd{WR-WRPC-MIB::wrpcPtpClockOffsetErrCnt} }
			\item [] \underline{Error condition}:\\
				{\footnotesize
				\texttt{wrpcPtpClockOffsetErrCnt != wrpcPtpClockOffsetErrCnt\_prev} }
      \item [] \underline{Action}:\\
		\end{pck_descr}

\subsubsection{\bf Detected jump in the RTT value calculated by \emph{PTP/PPSi}}
		\label{fail:timing:rtt_jump}
		\begin{pck_descr}
			\item [] \underline{Severity}: ERROR
			\item [] \underline{Mode}: \emph{Slave}
			\item [] \underline{Description}:\\
				Once a WR link is established the round-trip delay (RTT) can change
				smoothly due to the temperature variations. However, if a sudden jump is
				detected, that means that an erroneous timestamp was generated either on
				the Master or the Slave side.
				One cause of that could be the wrong value of the t24p transition point.
			\item [] \underline{SNMP objects}:\\
				{\footnotesize
				\snmpadd{WR-WRPC-MIB::wrpcPtpRTT}\\
				\snmpadd{WR-WRPC-MIB::wrpcPtpRTTErrCnt} }
			\item [] \underline{Error condition}:\\
				{\footnotesize
				\texttt{wrpcPtpRTTErrCnt != wrpcPtpRTTErrCnt\_prev} }
      \item [] \underline{Action}:\\
		\end{pck_descr}

\subsubsection{\bf Wrong $\Delta_{TXM}$, $\Delta_{RXM}$, $\Delta_{TXS}$,
		$\Delta_{RXS}$, $\alpha$ values are reported to the \emph{PTP/PPSi} daemon}
		\label{fail:timing:deltas_report}
		\begin{pck_descr}
			\item [] \underline{Severity}: ERROR
			\item [] \underline{Mode}: \emph{Slave}
			\item [] \underline{Description}:\\
				If \emph{PTP/PPSi} doesn't get the correct values of fixed hardware delays,
				it won't be able to calculate a proper Master-to-Slave delay. Although
				the estimated offset in \emph{PTP/PPSi} is close to 0, the WRS won't be
				synchronized to the Master with the sub-nanosecond accuracy.
			\item [] \underline{SNMP objects}:\\
				{\footnotesize
				\snmpadd{WR-WRPC-MIB::wrpcPtpDeltaTxM}\\
				\snmpadd{WR-WRPC-MIB::wrpcPtpDeltaRxM}\\
				\snmpadd{WR-WRPC-MIB::wrpcPtpDeltaTxS}\\
				\snmpadd{WR-WRPC-MIB::wrpcPtpDeltaRxS}\\
				\snmpadd{WR-WRPC-MIB::wrpcPtpAlpha} }
			\item [] \underline{Error condition}:\\
				{\footnotesize
				\texttt{wrpcPtpDeltaTxM == 0 || wrpcPtpDeltaRxM == 0 ||}\\
				\texttt{wrpcPtpDeltaTxS == 0 || wrpcPtpDeltaRxS == 0 ||}\\
				\texttt{wrpcPtpAlpha == 0} }
      \item [] \underline{Action}:\\
		\end{pck_descr}

\subsubsection{\bf PTP servo is not updating}
		\label{fail:timing:servo_not_updating}
		\begin{pck_descr}
			\item [] \underline{Severity}: ERROR
			\item [] \underline{Mode}: \emph{Slave}
			\item [] \underline{Description}:\\
				If PTP servo is not updating, we still increment the internal timing
				counters, but don't have updated information on the Master time and link
				delay. After some time the slave local time will drift away from the
				master.
			\item [] \underline{SNMP objects}:\\
				{\footnotesize
				\snmpadd{WR-WRPC-MIB::wrpcPtpServoUpdates}\\
				\snmpadd{WR-WRPC-MIB::wrpcPtpServoUpdateTime} }
			\item [] \underline{Error condition}:\\
				{\footnotesize
				\texttt{wrpcPtpServoUpdates != wrpcPtpServoUpdates} }
      \item [] \underline{Action}:\\
		\end{pck_descr}

\subsubsection{\bf \emph{SoftPLL} became unlocked}
		\label{fail:timing:spll_unlock}
		\begin{pck_descr}
			\item [] \underline{Severity}: ERROR / WARNING
			\item [] \underline{Mode}: \emph{all}
			\item [] \underline{Description}:\\
				If the \emph{SoftPLL} loses lock, for any reason, Slave, Master or Grand
				Master node can no longer be syntonized and phase aligned with its time
				source. WRPC in Master mode without properly locked Helper PLL is not
				able to perform reliable phase measurements for enhancing Rx timestamps
				resolution. For a Grand Master the reason of \emph{SoftPLL} going out of
				lock might be disconnected 1-PPS/10MHz signals or that the external
				clock is down.
			\item [] \underline{SNMP objects}:\\
				{\footnotesize
				\snmpadd{WR-WRPC-MIB::wrpcSpllMode}\\
				\snmpadd{WR-WRPC-MIB::wrpcSpllSeqState}\\
				\snmpadd{WR-WRPC-MIB::wrpcSpllAlignState}\\
				\snmpadd{WR-WRPC-MIB::wrpcSpllHlock}\\
				\snmpadd{WR-WRPC-MIB::wrpcSpllMlock}\\
				\snmpadd{WR-WRPC-MIB::wrpcSpllDelCnt} }
			\item [] \underline{Error condition}:\\
				{\footnotesize
        \texttt{wrpcSpllSeqState != ready\emph{(3)} ||}\\
        \texttt{[wrpcSpllMode == grandmaster\emph(1) \&\& wrpcAlignState != locked\emph{(6)}] ||}\\ % GrandMaster not locked
        \texttt{[wrpcSpllMode == slave\emph{(3)} \&\& wrpcSpllHlock == 0] ||}\\ % Spll slave and Hpll unlocked
				\texttt{[wrpcSpllMode == slave\emph{(3)} \&\& wrpcSpllMlock == 0] ||}\\ % Spll slave and Mpll unlocked
        \texttt{[wrpcSpllMode != grandmaster\emph{(1)} \&\& wrpcSpllMode != master\emph{(2)} \&\& wrpcSpllMode != slave\emph{(3)}]}} % Spll in neither of the GM/Master/Slave modes
			\item [] \underline{Warning condition}:\\
				{\footnotesize
        \texttt{[wrpcSpllMode == grandmaster\emph{(1)} \&\& wrpcSpllDelCnt > 0] ||}\\ % GrandMaster has unlocked from reference at some point
        \texttt{[wrpcSpllMode == master\emph{(2)} \&\& wrpcSpllDelCnt != wrpcSpllDelCnt\_prev] ||}\\ % Master just got unlocked
        \texttt{[wrpcSpllMode == slave\emph{(3)} \&\& wrpcSpllDelCnt != wrpcSpllDelCnt\_prev]} } % Slave just got unlocked
      \item [] \underline{Action}:\\
		\end{pck_descr}

\subsubsection{\bf WR link is down or FPGA not programmed or FPGA programmed with incorrect bitstream}
		\label{fail:timing:master_down}
		\begin{pck_descr}
			\item [] \underline{Severity}: ERROR
			\item [] \underline{Mode}: \emph{all}
			\item [] \underline{Description}:\\
				We monitor the WRPC over the WR network. We can realize if this only
				communication link is down by either SNMP requests timeouts or
				periodically pinging the device.
			\item [] \underline{SNMP objects}: \emph{(none)}
			\item [] \underline{Error condition}:\\
				{\footnotesize
				SNMP request timeout or PING timeout}
      \item [] \underline{Action}:\\
		\end{pck_descr}

\subsubsection{\bf PTP frames don't reach LM32}
		\label{fail:timing:no_frames}
		\begin{pck_descr}
			\item [] \underline{Severity}: ERROR
			\item [] \underline{Mode}: \emph{all}
			\item [] \underline{Description}:\\
				In this case, \emph{PTP/PPSi} will fail to stay synchronized and provide
				synchronization. Even if the WR servo is in the \texttt{TRACK\_PHASE}
				state, it calculates a new phase shift based on the Master-to-Slave delay
				variations. To calculate these variations, it still needs timestamped
				PTP frames flowing. There could be several causes of such fault:
				\begin{itemize}
					\item WR PTP Core HDL problem
					\item wrong VLANs configuration
				\end{itemize}
			\item [] \underline{SNMP objects}:\\
				{\footnotesize
				\snmpadd{WR-WRPC-MIB::wrpcPtpTx}\\
				\snmpadd{WR-WRPC-MIB::wrpcPtpRx}\\
				\snmpadd{WR-WRPC-MIB::wrpcPortInternalTx}\\
				\snmpadd{WR-WRPC-MIB::wrpcPortInternalRx} }
			\item [] \underline{Error condition}:\\
				{\footnotesize
				\texttt{wrpcPtpTx == wrpcPtpTx\_prev || wrpcPtpRx == wrpcPtpRx\_prev ||}\\
				\texttt{wrpcPortInternalTx == wrpcPortInternalTx\_prev ||}\\
				\texttt{wrpcPortInternalRx == wrpcPortInternalRx\_prev} }
      \item [] \underline{Action}:\\
		\end{pck_descr}

\subsubsection{\bf Detected SFP not supported for WR timing}
		\label{fail:timing:wrong_sfp}
		\begin{pck_descr}
			\item [] \underline{Severity}: ERROR
			\item [] \underline{Mode}: \emph{all}
			\item [] \underline{Description}:\\
				By not supported SFP for WR timing we mean a transceiver that doesn't
				have the \emph{alpha} parameter and fixed hardware delays defined in the
				SFP database. The consequence is \emph{PTP/PPSi} not having the right
				values to estimate link asymmetry. Despite \emph{PTP/PPSi} offset being
				close to 0 \emph{ps}, the device won't be properly synchronized.
			\item [] \underline{SNMP objects}:\\
				{\footnotesize
				\snmpadd{WR-WRPC-MIB::wrpcPortSfpPn}\\
				\snmpadd{WR-WRPC-MIB::wrpcPortSfpInDB} }
			\item [] \underline{Error condition}:\\
				{\footnotesize
        \texttt{wrpcPortSfpInDB != inDataBase\emph{(2)}} }
      \item [] \underline{Action}:\\
		\end{pck_descr}

\subsubsection{\bf SFP database not configured}
		\label{fail:timing:no_sfpdb}
		\begin{pck_descr}
			\item [] \underline{Severity}: ERROR
			\item [] \underline{Mode}: \emph{all}
			\item [] \underline{Description}:\\
				If there are no SFP entries in the database, any (even WR-supported) SFP
				cannot be matched with the calibration values for a given hardware and
				fiber. Despite \emph{PTP/PPSi} offset being close to 0 \emph{ps}, the
				device won't be properly synchronized.
			\item [] \underline{SNMP objects}:\\
				{\footnotesize
				\snmpadd{WR-WRPC-MIB::wrpcSfpDeltaTx.<n>}\\
				\snmpadd{WR-WRPC-MIB::wrpcSfpDeltaRx.<n>} }
			\item [] \underline{Note}: It's enough to try reading index 1 of the above
				SNMP objects tables to make sure there is at least one entry in the
				database.
			\item [] \underline{Error condition}:\\
				{\footnotesize
				Error when trying to get \texttt{wrpcSfpDeltaTx.1} and \texttt{wrpcSfpDeltaRx.1} SNMP objects}
      \item [] \underline{Action}:\\
		\end{pck_descr}

\newpage
\subsection{Other errors}
\label{sec:other_fail}

\subsubsection{\bf WR PTP Core reset}
		\label{fail:other:reset}
		\begin{pck_descr}
			\item [] \underline{Severity}: ERROR
			\item [] \underline{Description}:\\
				If the WRPC was reset it might either mean that there was a power cut or
				some not yet known bug caused the WRPC software to crash.
			\item [] \underline{SNMP objects}:\\
				{\footnotesize
				\snmpadd{WR-WRPC-MIB::wrpcTimeSystemUptime} }
			\item [] \underline{Error condition}:\\
				{\footnotesize
				\texttt{wrpcTimeSystemUpdate < wrpcTimeSystemUpdate\_prev} }
      \item [] \underline{Action}:\\
		\end{pck_descr}

\subsubsection{\bf WR PTP Core time reset}
		\label{fail:other:time_reset}
		\begin{pck_descr}
			\item [] \underline{Severity}: ERROR
			\item [] \underline{Description}:\\
				If the WRPC internal time counters are reset, this might mean the WR
				Master in the network has some problems and WRPC has followed the time
				reset. If that's not the case, this might mean some not yet known bug
				caused the WRPC time reset.
			\item [] \underline{SNMP objects}:\\
				{\footnotesize
				\snmpadd{WR-WRPC-MIB::wrpcTimeTAI}\\
				\snmpadd{WR-WRPC-MIB::wrpcTimeTAIString} }
			\item [] \underline{Error condition}:\\
				{\footnotesize
				\texttt{wrpcTimeTAI == 0} }
      \item [] \underline{Action}:\\
		\end{pck_descr}

\subsubsection{\bf Temperature of the node too high}
		\label{fail:other:temp}
		\begin{pck_descr}
			\item [] \underline{Severity}: WARNING
			\item [] \underline{Description}:\\
				If the temperature raises too high we might break our electronics. It
				also means that most probably something is wrong with the node cooling.
			\item [] \underline{SNMP objects}:\\
				{\footnotesize
				\snmpadd{WR-WRPC-MIB::wrpcTemperatureName.<n>}\\
				\snmpadd{WR-WRPC-MIB::wrpcTemperatureValue.<n>} }
			\item [] \underline{Error condition}:\\
				{\footnotesize
				\texttt{wrpcTemperatureValue.<n> > THRESHOLD} }
      \item [] \underline{Action}:\\
		\end{pck_descr}
